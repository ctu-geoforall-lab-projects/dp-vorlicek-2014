\documentclass[czech,11pt,a4paper]{article}
\usepackage[utf8]{inputenc}
\usepackage{a4wide}
\usepackage[pdftex,breaklinks=true,colorlinks=true,urlcolor=blue,
  pagecolor=black,linkcolor=black]{hyperref}
\usepackage[czech]{babel}

\pagestyle{empty}

\renewcommand{\arraystretch}{1.3}

\addtolength{\topmargin}{-.275in}

\begin{document}

\begin{center}
  {\Large --- Posudek vedoucího diplomové práce ---}
\end{center}

\vspace{.5cm}

\noindent \begin{tabular}{rp{.75\textwidth}}
  {\bf Diplomová práce:} & Prototyp turistického systému založeného na datech OpenStreetMap \\
  {\bf Student:} & Bc. Chrudoš Vorlíček \\
  {\bf Vedoucí:} & Ing. Martin Landa, Ph.D. \\
  {\bf Oponent:} & Ing. Daniel Bárta \\
\end{tabular}

\vspace{1cm}

Cílem diplomové práce Chrudoše Vorlíčka bylo navrhnout a implementovat
webovou turistickou aplikaci za použití dat OpenStreetMap. Jádrem
práce bylo propojení webové aplikace se sociální sítí Facebook.
\\

Text diplomové práce je členěn do pěti částí. V první části je uveden
stručný přehled podobných již existujících webových aplikací s důrazem
na motivaci této práce směřující především k propojení s populární
sociální sítí Facebook. Druhá a třetí část textu se dotýká popisu
technologií, které byly pro vývoj webové aplikace použity a také
projektu OpenStreetMap nad jehož daty je mapová aplikace postavena a
částečně s ním i propojena. Předposlední část textu je věnována
výstupu diplomové práce - tj. webové turistické aplikace propojené s
projektem OpenStreetMap a sociální sítí Facebook. V poslední kapitole
je výsledek práce zhodnocen a to především s ohledem na možný budoucí
vývoj projektu.

Text je po formální stránce uspokojující. V terminologii bohužel autor
sklouzává často k hovorovým výrazům, které by se v závěrečné práci
studenta technické univerzity objevovat neměly.
\\

Jako vedoucí práce mohu konstatovat, že spolupráce s diplomantem
probíhala hladce, na konzultace byl vždy dobře připraven a navržené
cíle práce plnil svědomitě. Výsledkem je prototyp webové aplikace,
která má, za předpokladu, že se projektu bude diplomant nadále
věnovat, šanci si najít své uživatele. Za největší přínos aplikace
považuji možnost přímo editovat data OpenStreetMap (v tomto případě
především zájmové body) a~její propojení se sociální sítí Facebook.
\\

Cíl diplomové práce byl podle mého názoru splněn. Práce v tomto ohledu
plně splňuje požadavky kladené na diplomovou práci na studijním
programu Geodézie a kartografie včetně všech formálních náležitostí.
\\

Diplomovou práci Chrudoše Vorlíčka doporučuji k obhajobě a hodnotím ji
stu\-pněm

\begin{center}
{\bf --- B (velmi dobře)  ---}
\end{center}

\vspace{1.2cm}

\noindent \begin{tabular}{lp{.28\textwidth}r}
V~Solanech dne 11.1.2014 & & \ldots\ldots\ldots\ldots\ldots\ldots\ldots \\
& & Ing. Martin Landa, Ph.D. \\
& & Fakulta stavební, ČVUT v Praze \\
\end{tabular}

\end{document}
